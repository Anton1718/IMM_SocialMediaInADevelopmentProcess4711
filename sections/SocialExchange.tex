\subsection{Toolkits f\"ur User Co-Design}
Firmen stellen den Konsumenten Toolkits zur Verf\"ugung, damit diese Produkte erstellen oder an ihre Bed\"urfnisse anpassen. Offenere Toolkits gehen dabei eher in erstere Richtung und setzen mehr auf User Innovation. Als Beispiel werden hier Kits zum Erstellen von Computer Chips genannt. Toolkits die den Nutzer nur eine Reihe von austauschbaren Modulen anbieten, gehen eher in die Richtung von User Co-Design und Anpassung. Der Dell Product Configurator ist ein Beispiel daf\"ur. Dadurch, dass die Benutzer es selbst herstellen, erh\"alt das Produkt einen Mehrwert f\"ur sie. Sie sind stolz auf das Ergebnis und m\"ochten es vielleicht auch anderen Leuten zeigen.\\
Toolkits k\"onnen allerdings auch Probleme verursachen. Falls die Auswahlm\"oglichkeiten zu komplex sind, ist der Nutzer eventuell \"uberfordert. Au\ss{}erdem fallen zus\"atzliche Kosten f\"ur Kundensupport an. Social Media kann diesen Problemen vielleicht entgegenwirken. Hilfe kann dadurch auch von Bekannten kommen, was auch hilft komplexe Entscheidungen zu vereinfachen.
Wenn die Produkte der Kunden durch Social Media allerdings mehr im Rampenlicht stehen, k\"onnen Kunden mit popul\"aren Designs auch Ruhm und Geld bekommen. Dadurch r\"uckt der Grund f\"ur die Benutzung eines Toolkits wiederum eher in die Richtung Monetary Incentive und somit die Methode Richtung Market Exchange.\\
Ein Beispiel f\"ur Toolkits w\"are die Webseite \url{www.mymuesli.com}, auf der man aus einer Auswahl von Zutaten w\"ahlen und sein eigenes Muesli kreieren kann. Da jedes Muesli eine ID bekommt, kann es auch von anderen Leuten bestellt werden. Es w\"are denkbar, dass ein Nutzer f\"ur die Schaffung eines beliebten, h\"aufig bestellten Produkts auch belohnt werden k\"onnte.\\
Ein anderes Beispiel w\"aren diverse PC Konfiguratoren, wie der von Ditech\cite{DITECH}. Der Kunde kann aus einer Menge von vorgefertigten PCs w\"ahlen und die eingebauten Teile seinen W\"unschen anpassen. Da sich viele Leute nicht so gut im Hardwarebereich auskennen, wird zus\"atzlich auch noch ein Support Service angeboten. W\"urde Ditech aber eine Einbindung in Facebook erlauben, k\"onnten sich auch Freunde den PC entwurf ansehen und Verbesserungsvorschl\"age machen. Diese w\"urde Ditech sicher Support Kosten ersparen.
\subsection{Lead-User Methode}
Im Paper wird ein Lead User als ein Konsument beschrieben, der bestimmte Bed\"urfnisse fr\"uher hat als der Rest des Marktes und stark vom Erf\"ullen dieser profitiert. Sie haben selbst innovative Ideen, wollen (oder k\"onnen) daraus jedoch keinen monet\"aren Profit machen. Dadurch teilen sie ihre Ideen mit Firmen oder andern Usern, was es den Firmen erm\"oglicht, sie f\"ur bestimmte innovative Probleme heranzuziehen. In solch einem Fall ist es auch nicht ungew\"ohnlich, dass Lead User andere vorschlagen, die sich besser mit einem Problem auskennen - Pyramiding.\\
Social Media macht es einfacher Lead User zu finden. Das hilft klarerweise Firmen, aber auch anderen Lead Usern, weil diese sich besser austauschen k\"onnen und Probleme effizienter l\"osen. Andererseits reduziert Social Media eventuell auch die Entrittsbarrieren des Marktes. Dadurch ist es Lead Usern eventuell doch einfacher m\"oglich mit ihrer Idee Profit zu machen, wodurch sie ein eigenes Unternehmen gr\"unden und nicht ihre Ideen weitergeben. So kann Social Media vielleicht auch zu einem h\"oheren Wettbewerb zwischen Lead Usern und Unternehmen f\"uhren. Weiters steigt auch der Wettbewerb um Lead User, wenn diese einfacher zu finden sind.













