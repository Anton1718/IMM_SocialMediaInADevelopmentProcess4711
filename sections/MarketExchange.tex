\subsection{Technical Solution Contest}
Ein Technical Solution Contest ist ein Wettbewerb bei dem ein Problem an eine gro\ss{}e Anzahl an Personen ausgesendet wird. Dies wird auch als Broadcasting bezeichnet. Die Personen die das Problem l\"osen sollen, werden als Solver bezeichnet und bekommen eine Belohnung wenn ihr L\"osungsvorschlag unter den besten ist. Um mehr Leute zu erreichen, und so an bessere L\"osungen zu kommen, wird oft eine Zwischenfirma beauftragt die sich auf die Suche nach eben diesen spezialisiert hat.
Social Media macht es nat\"urlich einfacher mehr Solver f\"ur ein Projekt zu erreichen. Das kann aber auch zum Nachteil der Zwischenfirma werden, denn wenn auch ohne sie eine gro\ss{}e Anzahl f\"ur den Wettbewerb gewonnen werden kann, wer braucht sie noch? Weiters \"andert sich auch das Verh\"altnis zwischen den Teilnehmern, welche sich jetzt gegenseitg beginnen kennenzulernen. So wird aus Einzelarbeit mit nur monet\"arer Belohnung Gruppenarbeit mit vermehrter sozialer Belohnung. Sollten sich die Teilnehmer dadurch gegenseitig positiv Beeinflussen und verstr\"arken kann das gut f\"ur den Zweck des Wettbewerbs sein. Sollte es dadurch allerdings vermehrt zu Trittbrettfahrern in Gruppen kommen und dadurch pro Person zu weniger L\"osungspotential kommen, kann das auch negativ zu bewerten sein.

\subsection{Ideation Contests}
Eine andere Art von Wettbewerb ist der Ideation Contest, der im Gegensatz zum Technical Solution Contest keine L\"osung f\"ur ein technisches Problem sucht, sondern Ideen oder Innovationen basierend auf einer vorgegebenen Aufgabe. Um zur Teilnahme zu motivieren gibt es Preise f\"ur die besten Ergebnisse. Da die Preise aber von eher geringerm Wert und die Anzahl der Teilnehmer hoch ist, spielen soziale Aspekte hier eine gr\"o\ss{}ere Rolle als beim Technical Solution Contest.
Wegen der starken \"ahnlichkeit zum  Technical Solution Contest ist die Verwendung von Social Media ungef\"ahr gleich einzusch\"atzen. Da die Kommunikation zwischen den Teilnehmern allerdings noch gemeinschaftlicher ist, k\"onnten Off-Topic Diskussionen in den vom Wettbewerbsveranstalter betriebenen Netzwerken und Foren ein schwerer zu regulierendes Problem werden.